\documentclass[12pt, a4paper]{ctexart}
\usepackage[margin=1in]{geometry}
\usepackage{xcolor}
\usepackage{tcolorbox}
\usepackage{enumitem}
\usepackage{setspace}
\usepackage{fancyhdr}

% 定义颜色
\definecolor{commentbox}{RGB}{245, 245, 245}
\definecolor{accentcolor}{RGB}{0, 102, 102} % 墨绿色
\definecolor{authorcolor}{RGB}{100, 100, 100}

% 设置页面风格
\pagestyle{fancy}
\fancyhf{}
\rhead{\small{经典夜读:张岱篇}}
\cfoot{\thepage}

\title{\textbf{\Huge 陶庵梦忆}}
\author{夜读选编}
\date{}

\begin{document}

\maketitle

\begin{center}
    \textit{\large “人无癖不可与交,以其无深情也;人无疵不可与交,以其无真气也。” —— 张岱}
\end{center}

\vspace{1cm}

% ==========================================
% 第一部分:湖心亭看雪
% ==========================================
\section{静之极:湖心亭看雪}

\begin{spacing}{1.2}
\large
崇祯五年十二月,余住西湖。大雪三日,湖中人鸟声俱绝。

是日更定矣,余拏一小舟,拥毳衣炉火,独往湖心亭看雪。雾凇沆砀,天与云与山与水,上下一白。湖上影子,惟长堤一痕、湖心亭一点、与余舟一芥、舟中人两三粒而已。

到亭上,有两人铺毡对坐,一童子烧酒炉正沸。见余大喜曰:“湖中焉得更有此人!”拉余同饮。余强饮三大白而别。问其姓氏,是金陵人,客此。

及下船,舟子喃喃曰:“莫说相公痴,更有痴似相公者!”
\end{spacing}

\begin{tcolorbox}[colback=commentbox, colframe=white, title=\textbf{【夜读微注】}]
\begin{itemize}[leftmargin=*]
    \item \textbf{人鸟声俱绝}:不仅是安静,而是声音完全消失,仿佛世界被静音了。
    \item \textbf{更定}:晚上八点左右。
    \item \textbf{毳(cuì)衣}:细毛皮衣。
    \item \textbf{雾凇沆砀(hàng dàng)}:形容冰花一片白茫茫的样子。
    \item \textbf{一痕、一点、一芥、两三粒}:这是极妙的数学视角。从线(痕)到点,再到微尘(芥、粒),在这个巨大的白色坐标系里,人变得极其渺小,但并不卑微。
\end{itemize}
\end{tcolorbox}

% ==========================================
% 第二部分:金山夜戏
% ==========================================
\section{动之极:金山夜戏}

\begin{spacing}{1.2}
\large
崇祯二年中秋后一日,余道镇江往兖州。日晡,至北固,舣舟江口。月光倒囊入水,江涛吞吐,露气吸之,噀天为白。余大惊喜,移舟过金山寺,已二鼓矣。经龙王堂,入大殿,皆漆黑。林下漏月光,疏疏如残雪。

余与其仆憨大,张灯亮板,纡朱怀金,在大殿中大铙大鼓,唱韩蕲王金山及长江大战诸剧。

锣鼓喧阗,一寺人皆起,看之向化,或是狐是鬼,不敢出。

剧毕,尚未曙,解缆过江。山僧至山脚,目送久之,不知是人、是怪、是鬼。
\end{spacing}

\begin{tcolorbox}[colback=commentbox, colframe=white, title=\textbf{【夜读微注】}]
\begin{itemize}[leftmargin=*]
    \item \textbf{日晡(bū)}:傍晚时分。
    \item \textbf{二鼓}:大约晚上九点到十一点。
    \item \textbf{噀(xùn)}:喷。月光像水一样被喷洒在天空中。
    \item \textbf{喧阗(tián)}:声音大得填满了空间。
    \item \textbf{看点}:想象一下,半夜两点,寂静的古庙突然灯火通明、锣鼓喧天,把和尚吓得以为见了鬼。这是古人的一场“行为艺术”。
\end{itemize}
\end{tcolorbox}

\vspace{1cm}

% ==========================================
% 第三部分:作者介绍
% ==========================================
\section{关于作者:张岱}

张岱(1597—1679),明末清初的文学家、史学家。他前半生是富家公子,爱繁华,爱美婢,爱奇珍异宝;后半生明朝灭亡,他国破家亡,躲进深山著书,虽然穷困潦倒,但精神依然高贵。

他就像一个时间的旅行者,用文字把那个逝去的、精致的梦境(《陶庵梦忆》)封存了起来。他的文章短小精悍,却有着极高的“精神密度”。

% ==========================================
% 第四部分:品评与典故(时空对话)
% ==========================================
\section{品评:中国文人的精神坐标}

读张岱,我们不仅是在读明朝,更是在读中国文人骨子里流传千年的一种“气”。

张岱在《湖心亭》里冷得像冰,在《金山寺》里狂得像火。这看似矛盾,其实是一个灵魂的两面:因为对世界极其深情,所以要么彻底沉静,要么彻底释放。

这种精神,最早可以追溯到魏晋时期的“竹林七贤”。张岱的行为,其实是对两位前辈的致敬:

\subsection*{1. 面对绝望的深情:阮籍之哭}
当你在《金山夜戏》里看到张岱半夜唱戏的大笑时,你要想起那个因为无路可走而大哭的阮籍。

\begin{quote}
\textbf{典故原文(选自《晋书·阮籍传》):}\\
“(阮籍)时率意独驾,不由径路,\textbf{车迹所穷,辄恸哭而反}。”
\end{quote}

\noindent\textbf{【解读】}:阮籍驾车乱走,路断了就对着荒野大哭。他哭的不是路不通,而是一种对命运自身的表达。张岱的狂笑与阮籍的痛哭,本质上都是一种不妥协。

\vspace{0.5cm}

\subsection*{2. 面对美好的洒脱:王徽之访戴}
当你在《湖心亭看雪》里看到张岱痴迷于那一点影子时,你要想起那个雪夜访友却不进门的王徽之。

\begin{quote}
\textbf{典故原文(选自《世说新语·任诞》):}\\
“(王徽之)夜大雪... 忽忆戴安道。时戴在剡,即便夜乘小船就之。经宿方至,造门不前而返。人问其故,王曰:‘\textbf{吾本乘兴而来,兴尽而返,何必见戴?}’”
\end{quote}

\noindent\textbf{【解读】}:对于他们来说,“兴致”比“结果”重要。
\begin{itemize}
    \item 王徽之花了一夜坐船,享受了“访友”的过程,这就够了,见不见面无所谓。
    \item 张岱冒着严寒去湖心亭,享受了“独处”的过程,这就够了,是不是疯子无所谓。
\end{itemize}

\vspace{1cm}
\hrule
\vspace{0.5cm}

\textbf{给小朋友的话:}\\
无论是张岱、阮籍还是王徽之,他们都告诉我们一个道理:在这个世界上,你可以不按常理出牌,可以有点怪,只要你的内心是真诚的、有趣的。做一个心灵自由的人,远比做一个循规蹈矩的人更快乐。

\end{document}