\documentclass[12pt, a4paper]{ctexart}
\usepackage[margin=1in]{geometry}
\usepackage{xcolor}
\usepackage{tcolorbox}
\usepackage{enumitem}
\usepackage{setspace}
\usepackage{fancyhdr}
\usepackage{titlesec}

% 定义颜色风格
\definecolor{stormcolor}{RGB}{47, 79, 79} % 深岩灰色 (Dark Slate Gray)
\definecolor{starcolor}{RGB}{25, 25, 112} %不仅蓝 (Midnight Blue)
\definecolor{commentbox}{RGB}{248, 248, 255}

% 设置页面风格
\pagestyle{fancy}
\fancyhf{}
\rhead{\small{夜读:天地双响}}
\cfoot{\thepage}

\title{\textbf{\Huge 自然的呼吸}}
\author{John Muir \ \& \ Alphonse Daudet}
\date{}

\begin{document}

\maketitle

\begin{center}
    \large
    今晚,我们先去加利福尼亚的森林里听一场**重金属摇滚**,\\
    再去普罗旺斯的山顶听一首**摇篮曲**。
\end{center}

\vspace{1cm}

\section*{\textcolor{starcolor}{第二乐章:繁星的婚途}}
\textit{选自 Alphonse Daudet《磨坊书简》(Lettres de mon moulin)。年轻的牧羊人独自在吕贝隆山顶牧羊,有一天,他暗恋的主人小姐因为迷路,被迫要在山上过夜。}

\begin{spacing}{1.3}
\large
\textbf{繁 星}
\textbf{—— 普罗旺斯牧羊人的故事}

当我在吕贝隆(Luberon)山上放牧的时候,经常一连好几个星期都待在牧场里,除了我的牧羊犬“拉布里”和我的羊群,见不到一个活人。

偶尔,于尔山(Mont-de-l'Ure)的隐士路过这里寻找草药,或者我会看到那个来自皮埃蒙特的烟囱清扫工,脸涂得黑黑的。但这些都是淳朴得有些木讷的人,孤独让他们变得沉默寡言,他们早已失去了闲聊的兴致,对山下村镇里发生的事一无所知。

所以,我是多么快乐啊!每隔半个月,当我听到山路上那是给我们农场送补给的骡子的铃声,当我看到小农场工那张明亮的小脸,或是诺拉德大婶的红帽子从山丘后面冒出来时,我简直高兴坏了。我会缠着他们问村里的新闻:谁家受洗了?谁家结婚了?

但我最关心的,是想知道我的主人的女儿——斯蒂芬妮特(Stephanette)小姐的消息。她是方圆五十公里内最美的人儿。我不想显得太八卦,但总是忍不住旁敲侧击:她有没有去参加村里的节日庆典?有没有去晚上的聚会?是不是每次都有新的追求者围着她转?

如果有人问我,你一个穷山沟里的牧羊人,管这些闲事干什么?我会告诉他:那年我才二十岁,而斯蒂芬妮特是我这辈子见过的最美的事物。

然而,有一个周日,半个月一次的补给迟迟未到。
上午我想:“大概是因为做大弥撒耽误了吧。”
到了中午,一场大暴雨袭来,我又想:“大概是路不好走,骡子没法出发。”

直到下午三点左右,天空放晴,湿润的山峦在阳光下闪闪发光。就在那时,在树叶滴水声和溪流的咆哮声中,我终于听到了骡子的铃声。对我来说,那声音就像复活节的钟声一样欢快、活泼。

但是,走在前面的既不是小农场工,也不是诺拉德大婶。而是……你们怎么也猜不到的……朋友们,那是\textbf{我心中的渴望}!竟然是斯蒂芬妮特小姐本人!她端坐在柳条筐之间,山里的空气和刚才的风暴让她美丽的脸庞泛着红晕。

原来,小农场工病了,诺拉德大婶去孩子家度假了。斯蒂芬妮特一边下骡子一边告诉我这些,还解释说她来晚是因为迷路了。

但是,看着她穿着节日的盛装——戴着花饰缎带,穿着丝绸裙子和蕾丝紧身胸衣——与其说她是刚从灌木丛里找路过来,倒更像是刚从一场舞会上回来。噢,这个可爱的小人儿!我看她怎么也看不够。我从来没有离她这么近过。以前在冬天,羊群回到平原,我也只能在晚上吃晚饭时见她一面。她总是穿得花枝招展,有点骄傲,匆匆穿过餐厅,几乎不正眼看我们……

而现在,她就在那儿,就在我面前,只属于我一个人。这难道不值得让人神魂颠倒吗?

斯蒂芬妮特把补给从筐里拿出来后,开始对一切都感到好奇。她稍微提起那条漂亮的周日长裙(怕被弄脏),走进围栏,查看我睡觉的地方。那个铺着羊皮的稻草窝、我挂在墙上的长斗篷、我的牧羊杖,还有我的弹弓……所有这些东西都让她着迷。

“所以,这就是你住的地方,我的小牧羊人?一直一个人待着,一定很无聊吧?你平时都干些什么?都在想些什么呢?”

我很想说:“想你啊,我的小姐。”而且我没撒谎。但我太窘迫了,一句话也答不上来。她显然看出了这一点,于是带着一点点可爱的恶作剧心理,故意逗我:

“牧羊人,你的女朋友不来看看你吗?那是当然,她一定是那个传说中的‘埃斯特雷尔仙女’吧?只有她才会在山顶上奔跑……”

她说话的样子,在我眼里就像真的埃斯特雷尔仙女。她仰起头,发出一串调皮的笑声,然后匆匆离开了。她的来访就像一场梦。

“再见,牧羊人。”
“再见,再见,小姐。”

她走了,带着空篮子消失了。

看着她在陡峭的山路上消失,骡子蹄下踢落的石子滚下山坡,仿佛也带走了我的心。我听着那石子滚落的声音,听了很久很久。直到傍晚,山谷深处变成了深蓝色,羊群咩咩叫着聚拢回来,我还在那里发呆,不敢动弹,生怕打破那个梦境。

就在这时,我听到下坡处有人喊我。
是她!斯蒂芬妮特小姐又回来了。

但这一次,她不再笑得花枝乱颤。她浑身湿透,瑟瑟发抖,惊魂未定。原来在山脚下,索尔格河(River Sorgue)因为暴雨涨水了,她想强行渡河,差点淹死。最糟糕的是,天已经黑了,她不可能回农场了,她找不到路,而我也不能离开羊群送她。

想到要在山上过夜,她非常不安,更担心家里人会着急。我尽力安慰她:
“七月的夜很短,小姐。这也只是一会儿的不方便。”

我赶紧生了一堆旺火,帮她烘干湿透的脚和裙子。我端来牛奶和奶酪,但这个可怜的小东西既没心思烤火,也没胃口吃东西。看着大颗大颗的眼泪在她眼眶里打转,我也想跟着哭了。

夜幕降临了。山脊上只剩下最后一抹夕阳的余晖。我想让小姐去围栏里的草窝休息。我在新鲜的稻草上铺了一张崭新的皮毛,祝她晚安。

我打算坐在门外守着。上帝作证,尽管我心中有一团火在烧,但我没有一丝邪念。我只感到一种巨大的骄傲——在这个围栏的角落里,在好奇的羊群注视下,\textbf{我的主人的女儿,正像一只羊一样——一只比所有羊都更洁白、更珍贵的羊——在我的守护下安睡。}

对我来说,天空从未如此深邃,星星从未如此明亮。

突然,围栏的门开了,美丽的斯蒂芬妮特走了出来。她睡不着。羊群在梦中吃草、咩咩叫的声音吵到了她。她想离火近一点。

我把我的山羊皮披在她肩上,拨旺了火。我们就这样并肩坐着,什么也没说。

如果你曾在星空下露宿,你一定知道:在我们睡觉的时候,一个神秘的世界会在孤独和寂静中苏醒。那是泉水歌唱得更清晰的时候,是池塘里点起鬼火的时候。所有的山林精灵都自由地游荡,空气中有各种沙沙声,那是树枝在变粗、草叶在生长的声音。

白天属于日常的生物,而夜晚属于奇异的、未知的事物。如果你不习惯,它会让你感到恐惧。

小姐就是这样。她浑身战栗,一有风吹草动就紧紧抓着我。突然,从黑暗的池塘深处传来一声长长的、凄厉的叫声,声音起伏着向我们逼近。与此同时,一颗流星在头顶划过,仿佛刚才那声哀鸣不仅带着声音,还带着光。

“那是什么?”斯蒂芬妮特低声问。
“是一个灵魂正在进入天堂,小姐。”我划了个十字。

她也跟着划了十字,然后仰望着天空,神情入迷。过了一会儿,她问我:
“牧羊人,这是真的吗?你们这行的人都是魔法师?”
“不,不,小姐。但我们住在离星星更近的地方,我们比平原上的人更了解上面发生了什么。”

她依然盯着星星,双手托着头,裹着羊皮,像一个小小的牧羊人。
“星星真多啊!真美!我从来没见过这么多。你知道它们的名字吗?”
“当然,小姐。你看!就在我们头顶,那是‘圣雅克路’(银河)。再那边是大熊星座……”

我详细地给她描述星空的魔法。
“还有一颗星,我们牧羊人叫它‘玛格洛娜’,”我说,“它追逐着土星,每隔七年就会和它结婚。”
“什么?牧羊人!星星也会结婚吗?”
“噢,是的,小姐。”

我正想向她解释这婚礼是怎么回事,突然感到肩膀上有什么凉凉的、柔软的东西。

那是她沉甸甸的头,带着睡意,带着丝带、蕾丝和波浪般的长发,轻轻地靠在了我的身上。

她就这样一动不动,直到群星在黎明中渐渐隐去。

我就这样看着她沉睡,虽然内心深处有些许慌乱,但这清澈的夜只给了我美好的念头,让我的心始终保持纯洁。

在我们四周,群星继续着它们庄严、无声的旅程,就像天空中一大群温顺的羊群。

在那一刻,我甚至以为:\textbf{在那无数闪烁的星星中,有一颗最精致、最明亮、最美丽的星星,因为它迷了路,所以从天空中滑落下来,正好停在了我的肩膀上,睡着了……}

\end{spacing}

\begin{tcolorbox}[colback=commentbox, colframe=starcolor, title=\textbf{【陪练点拨】}]
从缪尔的喧嚣进入都德的寂静。这里的核心意象是\textbf{“归位”}。
\begin{itemize}
    \item 所有的星星都在轨道上行走(天体的秩序)。
    \item 迷路的小姐最后靠在牧羊人肩头睡着了(人的秩序)。
    \item 最后那句比喻,把宏大的宇宙(星)和微小的人(小姐)合二为一。这是一场无声的、神圣的婚礼。
\end{itemize}
\end{tcolorbox}

\end{document}