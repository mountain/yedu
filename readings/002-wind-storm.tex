\documentclass[12pt, a4paper]{ctexart}
\usepackage[margin=1in]{geometry}
\usepackage{xcolor}
\usepackage{tcolorbox}
\usepackage{enumitem}
\usepackage{setspace}
\usepackage{fancyhdr}
\usepackage{titlesec}

% 定义颜色风格
\definecolor{stormcolor}{RGB}{47, 79, 79} % 深岩灰色 (Dark Slate Gray)
\definecolor{biocolor}{RGB}{85, 107, 47} % 橄榄绿
\definecolor{philocolor}{RGB}{70, 130, 180} % 钢蓝
\definecolor{commentbox}{RGB}{250, 250, 250}

% 设置标题格式,避免正文紧贴标题
\titlespacing*{\section}{0pt}{1.5ex plus 1ex minus .2ex}{1ex plus .2ex}

% 设置页面风格
\pagestyle{fancy}
\fancyhf{}
\rhead{\small{经典夜读:地籁与松涛}}
\cfoot{\thepage}

\title{\textbf{\Huge 自然的呼吸}}
\author{John Muir \ \& \ 庄子}
\date{}

\begin{document}

\maketitle

\begin{center}
    \large
    \textit{今晚,我们先去加利福尼亚的森林里听一场\textbf{重金属摇滚},\\
    再回到两千多年前的中国,听一首\textbf{三声部的哲学赋格}。}
\end{center}

\vspace{0.8cm}

% ==========================================
% 第一部分:John Muir 正文
% ==========================================
\section*{\textcolor{stormcolor}{森林里的风暴}}
\textit{选自 John Muir《群山之间》(The Mountains of California)}

\begin{spacing}{1.2} % 稍微收紧一点行距,让长文更紧凑
\large

山间的风,就像露水与雨水、阳光与雪一样,是经过度量并带着爱意赐予森林的,旨在培育它们的力量与美。虽然其他影响森林的因素可能受到地域的限制,但风的影响却是普遍的。雪在每个冬天压弯并修剪高处的森林;闪电偶尔劈中零星的树木;雪崩则像园丁修剪花坛一样,瞬间扫平数千棵树。但唯有风,它走向每一棵树,抚弄每一片叶子、每一根枝条和每一道皱裂的树干;没有一棵被遗忘。无论是在冰峰崎岖壁垒上张开双臂耸立的山地松,还是在幽谷中最低调隐居的小树,风都会寻找并找到它们。风温柔地爱抚它们,弯曲它们进行激烈的锻炼,刺激它们生长,按需摘去一片叶或一根枝,或者移走整棵树乃至整片树林。风时而像睡眼惺忪的孩子在枝头低语呢喃,时而像海洋一样咆哮。风祝福着森林,森林也祝福着风,其必然结果便是那无法言喻的美丽与和谐。

当一个人看到直径六英尺的巨松在山地大风面前像野草一样弯腰,并不时看到巨树轰然倒下、震撼山岳时,他会感到惊讶:除了那些最低矮、最粗壮的树木外,竟然还有树能找到一段足够平静的时间来扎根;或者一旦扎根,竟然没有迟早被吹倒。但是,当风暴过去,我们看到同样的森林重归宁静,依然挺拔、清新、毫发无损地巍然耸立;当我们想到自它们种下以来,已经承受了几个世纪的风暴——冰雹打断嫩苗,闪电烧灼劈裂,大雪、狂风和雪崩碾压覆盖——而这一切狂野的“风暴栽培”的显著结果,竟是我们眼前这光辉的完美时;那么,我们对“大自然的林学”的信心便建立起来了,我们不再抱怨那些毁灭性大风或任何其他风暴工具的暴烈。

在内华达山脉(Sierra)的森林里,有两种树只要保持健康,就永远不会被吹倒。它们是生长在峰顶的刺柏(Juniper)和矮松(Dwarf Pine)。它们坚硬弯曲的根系像鹰爪一样死死抓进饱受风蚀的岩层,而它们柔韧如绳索般的枝条则顺从地弯曲,无论风多猛烈,受力面积都很小。其他的高山针叶树——如白皮松(Needle Pine)、山地松、扭叶松(Two-leaved Pine)和铁杉(Hemlock Spruce)——也从未因风暴而被破坏性地稀疏,这得益于它们令人赞叹的韧性和紧密的生长方式。

至于低海拔地区的巨树们,情况也大致如此。王者般的糖松(Sugar Pine),高耸入云达200英尺,是风暴的绝佳靶子;但它的叶子并不茂密,它长长的水平长臂在狂风中顺从地旋转,就像溪流中飘动的绿色水藻发辫;而银冷杉(Silver Firs)在大多数地方都紧密地排列在一起,以此获得团结的力量。黄松(Yellow Pine 或 Silver Pine)比内华达山脉上的任何其他树都更容易被吹倒,因为它的枝叶相对于高度来说形成了一个更大的受力团块,而且它们往往在许多地方生长稀疏,留出了开阔的通道,让风暴可以长驱直入。此外,由于它分布在山脉的低处——即冰川冬季结束时最早随着冰盖破裂而裸露出来的区域——其生长的土壤暴露在冰川后风化作用下的时间更长,因此比较高处的土壤更加松散、腐朽,无法为根系提供稳固的锚定。

在探索沙斯塔山(Mount Shasta)的林带时,我发现了一条飓风扫过的路径,上面布满了数千棵这种松树的残骸。无论大小,它们都被纯粹的暴力连根拔起或拧断,留下了一道干干净净的缺口,就像雪崩造成的一样。但在内华达山脉,能造成这种破坏的飓风极为罕见。当我们从山脉的一端探索到另一端时,无论我们如何看待那些造就了森林的自然力量,我们都不得不相信:这些森林是地球表面最美丽的森林。

不仅是林中风声让人深感兴奋——它或多或少能影响每一个人的心智——树木(尤其是针叶树)在风中表现出的那种如水般的流动(waterlike flow)也同样迷人。没有任何其他树木能像它们那样,将风表现得如此广阔和具象,即使是热带那对微风最敏感的棕榈或树蕨也无法相比。巨杉(Sequoias)森林的挥舞令人难以形容地印象深刻且崇高,但在我看来,松树才是风最好的翻译官。它们是挥舞着的巨大金色权杖,永远在调子上,在它们长达百年的生命中不停地歌唱和书写着风的音乐。

然而,在严格的高山林区,你很少能看到或听到这种高贵的树之挥舞和树之音乐。粗壮的刺柏,其周长有时超过高度,几乎和它生长的岩石一样僵硬。矮松那细长鞭状的枝条会在波纹中颤动,但即便是其中最高最细的,也太不屈了,哪怕在最猛烈的狂风中也不会挥舞,它们只会在急促的短颤中抖动。不过,铁杉、山地松以及一些最高的扭叶松丛,在风暴中会以相当大的幅度和优雅的姿态鞠躬。但是,只有在中低海拔地带,风与树的相遇才展现出所有的壮丽。

我曾在内华达山脉经历过一场最美丽、最令人振奋的风暴,那是1874年12月,当时我碰巧在探索尤巴河(Yuba River)的一条支流河谷。天空、地面和树木都被雨水彻底冲洗过,又变干了。那是一天极度纯净的日子,是加州冬天无可比拟的时刻之一——温暖、芬芳,充满白色的闪耀阳光,洋溢着春日最纯净的气息,同时又被一场你可以想象的最令人振奋的风暴所激活。我当时没有像往常一样露营,而是借住在一位朋友家里。但是,当风暴开始奏响时,我立刻冲进了森林去享受它。因为在这种时刻,大自然总有稀罕的东西要展示给我们,而那点生命危险,比起像个懦夫一样缩在屋顶下,根本算不了什么。

当我完全沉浸其中时,还是清晨。可爱的阳光倾泻在山丘上,照亮了松树的顶端,释放出一股夏日般的芬芳,这与风暴狂野的基调形成了奇异的对比。空气中斑驳地飞舞着松树的流苏和亮绿色的羽状枝叶,它们像被追逐的鸟儿一样在阳光中闪过。但空气里没有一丝灰尘,没有比树叶、成熟的花粉、干枯的蕨类和苔藓微粒更不纯净的东西。几个小时里,我听到树木以每两三分钟一棵的速度倒下;有些因为地面松软浸水而被连根拔起;有些则因为火灾留下的弱点而被拦腰折断。

各种树木的姿态构成了一项令人愉快的力学研究。年轻的糖松,轻盈得像松鼠尾巴,几乎鞠躬到了地面;而那些古老的族长级巨松,其巨大的树干经受过百次风暴的考验,庄严地在上方挥舞,它们长长的、拱形的枝条在狂风中流畅地飘动,每一根针叶都在颤动、鸣响,像钻石一样折射出利剑般的光芒。道格拉斯黄杉(Douglas Spruces,即花旗松)长着长长的发辫状枝条,针叶聚集成灰色的、闪烁的光晕,它们伫立在山顶,轮廓鲜明,景象惊人。幽谷中的浆果鹃(Madroños),红色的树皮和巨大的光泽叶片向各个方向倾斜,反射着阳光,闪烁着颤动的光芒,就像人们常在冰川湖的波纹表面看到的那样。

但此时,银松(Silver Pines)是最令人印象深刻的。200英尺高的巨塔像柔韧的金色权杖一样挥舞、深深鞠躬,仿佛在崇拜,而它们长长的、颤抖的枝叶整体被点燃成一片连绵的白色烈焰。狂风的力量如此之大,以至于其中最稳固的君王也摇晃到了根部——如果你靠在树干上,能明显感觉到这种运动。大自然正在举行盛大的节日,即使是最僵硬的巨树,身上的每一根纤维也都因快乐的兴奋而颤栗。

我在这种激情的音乐和运动中漂流,穿过许多幽谷,从一个山脊到另一个山脊;经常停在岩石的背风处躲避,或是凝视、倾听。即使当这首宏大的赞美诗奏响到最高音时,我依然能清晰地分辨出不同树木的音调——云杉、冷杉、松树和无叶的橡树——甚至脚下枯草发出的无限温柔的沙沙声。每一棵树都在以自己的方式表达,唱着自己的歌,做着自己独特的姿态——这种丰富多样的变化是我在其他森林中从未见过的。加拿大、卡罗来纳和佛罗里达的针叶林由彼此相似的树木组成,就像草叶一样,生长方式也大同小异。一般来说,针叶树很少拥有像橡树和榆树那样鲜明的个性。但加州的森林比世界上任何其他森林都包含更多不同的物种。在这里,我们不仅能发现明显的种群分化,而且几乎每一棵树都有鲜明的个性,这使得风暴的效果呈现出不可描述的辉煌。

中午时分,在穿过榛子树和美洲茶(Ceanothus)灌木丛进行了一段漫长而刺痛的攀爬后,我到达了附近最高山脊的顶端。突然,我想到了一个主意:爬上一棵树去获得更开阔的视野,并把耳朵贴近它最高处针叶发出的“风神竖琴”(Aeolian music)般的音乐,那该是件多美妙的事。但在这种情况下,选择一棵树是件严肃的事。有些树根基不稳,似乎有被吹倒的危险,或者可能被其他倒下的树砸中;有些树在离地很高的地方都没有树枝,同时树干又太粗,手脚无法环抱攀爬;而其他的树位置不佳,视野不开阔。经过谨慎的观察,我选中了一组像草丛一样紧挨着生长的道格拉斯黄杉中最高的一棵,看起来除非这组树全倒,否则它不会单独倒下。虽然相对年轻,但这棵树也有约100英尺高,它那柔韧的、毛茸茸的树梢正在狂喜中剧烈摇摆和旋转。我有丰富的植物学爬树经验,所以没费什么劲就爬到了这棵树的顶端。我从未体验过如此高贵的运动快感!细长的树梢在激情的洪流中拍打、呼啸,前后弯曲,一圈圈地旋转,画出无法描述的垂直和水平曲线组合,而我紧紧抓住树干,肌肉紧绷,就像芦苇荡里的一只食米鸟(Bobolink)。

在摆动幅度最大时,我的树梢划出了20到30度的弧线。但我对它的弹性充满信心,因为我曾见过同种的其他树木经受过更严酷的考验——在厚重的积雪中甚至弯到了地面——却连一根纤维都没断。因此我是安全的,我可以自由地让风融入我的脉搏,从这个绝佳的瞭望点享受这片兴奋的森林。无论在什么天气,这里的景色都必定极其美丽。此刻,我的目光扫过松林覆盖的山丘和谷地,就像扫过一片片起伏的麦田。随着闪亮的枝叶被相应的气浪搅动,我感觉到光线像波纹和宽阔的涌浪一样,从一个山脊跨越山谷流向另一个山脊。这些反射光的波浪经常会突然破碎成一种被拍打的泡沫;有时,它们又有序地相互追逐,似乎以前倾的同心曲线向前弯曲,然后像拍击在倾斜海岸上的海浪一样消失在某个山坡上。弯曲的针叶反射的光量是如此之大,以至于整片树林看起来像被雪覆盖了一样,而树下黑色的阴影极大地增强了这种银色光辉的效果。

除了阴影之外,这片狂野的松林之海中没有任何阴郁的东西。相反,尽管这是冬季,色彩却异常美丽。松树和肖楠(Libocedrus)的树干呈棕色和紫色,大部分枝叶都染上了黄色;月桂树丛翻起叶片浅色的背面,形成团团灰色;还有来自熊果丛(Manzanita)的一抹抹巧克力色,以及浆果鹃(Madroños)树皮上喷薄而出的鲜艳绯红;而山坡上的地面,透过树丛间的空隙显露出来,呈现出大片的淡紫色和棕色。

风暴的声音与这狂野的光影运动完美对应。裸露枝干发出的深沉低音,像瀑布一样轰鸣;松针发出的急促、紧张的振动,时而升为尖锐的呼啸,时而降为丝绸般的低语;幽谷中月桂树林的沙沙声,以及叶片撞击发出的敏锐的金属咔哒声——只要静心倾听,这一切都很容易分析辨别。

万物的姿态变化被展现得淋漓尽致,以至于人们仅凭姿态,或者结合它们的形态、颜色以及反射光线的方式,就能在数英里外辨认出不同的树种。所有的树看起来都强壮而舒适,仿佛真心享受着风暴,并响应着它最热情的问候。如今我们常听到关于“生存竞争”的说法,但在这里,我看不到任何世俗意义上的“挣扎”;没有一棵树表现出对危险的认知;没有祈求;相反,只有一种不可战胜的快乐,这种快乐既远离狂喜,也远离恐惧。

我在高处待了几个小时,时常闭上眼睛,独自享受音乐,或安静地享用那流淌而过的芬芳。森林的香气不如暖雨时节那么明显(那时无数香脂般的芽和叶像茶一样被浸泡),但是,由于树脂枝条的相互摩擦和无数针叶的不断磨损,这场大风被调味到了非常提神的程度。而且,除了这些本地的香气,还有从远方带来的气味踪迹。因为这风先是来自大海,摩擦过它新鲜咸涩的波浪;然后经过红木林,穿过长满蕨类植物的深谷,漫过海岸山脉许多开满鲜花的山脊,铺展成宽阔起伏的气流,接着越过金色的平原,爬上紫色的山麓,最后带着一路收集的各种香薰,进入了这片松林。

风是它所触碰过的一切事物的广告,无论我们能解读多少;仅凭气味,它们就在讲述自己的流浪历程。水手能在远离陆地的海上闻到陆风中的花香,而海风也会将红藻和海带的香气带到内陆深处,虽然混合着千种陆地花香,却能被迅速辨认出来。为了说明这一点,我可以讲个故事:我小时候在苏格兰的福斯湾(Firth of Forth)呼吸过海风;后来被带到威斯康星州,在那里待了十九年;在这期间我没有呼吸过一口海风。后来,我独自一人从中西部的密西西比河谷步行到墨西哥湾进行植物学考察。当我在佛罗里达,远离海岸,全神贯注于身边壮丽的热带植被时,我突然辨认出了一股海风——它正从棕榈树和开花的藤蔓纠缠中渗透过来。这气味瞬间唤醒并释放了一千个休眠的联想,让我变回了苏格兰的那个男孩,仿佛中间隔着的所有岁月都被湮灭了。

大多数人喜欢看山间的河流,并把它们记在心里;但很少有人在乎去看风,尽管风更加美丽崇高,而且有时它们像流水一样清晰可见。当冬天的北风扫过内华达高山的弧形峰顶时,这事实有时会通过绵延一英里的飞雪旗帜公之于众。那些以此种方式显形的风,即便是最迟钝的想象力也能看见。而在动荡的森林中环顾四周,我们可以通过风对树木的影响看到风的样子。在那边,它像急流的波纹一样冲下来,扫过弯曲的松树,从一座山头到另一座山头。近处,我们看到被吹落的羽状枝叶,时而在水平气流中飞驰,时而在漩涡中打转,或是从漩涡边缘逃逸,在巨大的上升气穹上高飞,或在火焰般的浪尖上翻腾。平滑的深层气流、瀑布、跌水和旋转的涡流,在每一棵树、每一片叶子周围歌唱,并随着该地区多变的地形改变形态,就像山间河流顺应河道特征一样。

在追踪内华达山脉的溪流从源头到平原的过程中——看着它们在瀑布中绽放洁白,在水晶般的激流中滑行,在巨石阻塞的峡谷中泛着灰色泡沫汹涌澎湃,在森林中流淌成长长的宁静河段——在详细了解了它们的语言和形式之后,我们最终可能会听到它们汇成一首宏大的赞美诗,并在清晰的内心视野中理解它们,看着它们像蕾丝一样覆盖山脉。但是,即便是这一壮观景象,也不如我们在山地森林中看到的这些空气风暴之流(storm-streams of air)那样崇高,也不比其实质性多出一丝一毫。

我们都在银河中一起旅行,树木和人;但直到这个风暴日,当我挂在风中摇摆时,我才从未想过:在通常意义上,树木也是旅行者。它们进行许多次旅行,虽不是长途跋涉;但我们自己的一去一回的小小旅程,比起树木的挥舞,其实也多不了多少——甚至很多还不如它们。

当风暴开始减弱,我从树上下来,漫步穿过渐渐平静的森林。风暴的音调消逝了。我转向东方,看到无数森林大军在山坡上肃立,安详而宁静,层层叠叠,就像一群虔诚的听众。夕阳给它们披上了琥珀色的光辉,仿佛一边看着它们倾听,一边说:“我将我的平安赐给你们。”

凝视着这令人印象深刻的景象,所有所谓的风暴破坏都被遗忘,这些高贵的森林从未显得如此清新、如此快乐、如此不朽。

\end{spacing}

\begin{tcolorbox}[colback=commentbox, colframe=biocolor, title=\textbf{约翰·缪尔 (John Muir, 1838-1914)}]
\begin{spacing}{1.2}
他被称为“国家公园之父”,但他首先是一个\textbf{发明家和工程师}。

年轻时,缪尔擅长制造精密的木制时钟和自动化机器。直到一次工厂事故差点让他失明,恢复视力后,他决定把余生都用来“看”这个世界。他扔掉了发明的机器,走进了荒野。

正因为有着工程师的背景,他看大自然的眼光与众不同。别人看森林是木材,是风景;他看森林是结构,是流体力学,是精密运转的系统。他为了观察风暴敢爬上30米高的树顶,为了研究冰川敢在悬崖边过夜。他用这种“硬核”的方式告诉世人:\textbf{荒野不是混乱的,而是神圣的秩序。}
\end{spacing}
\end{tcolorbox}

\newpage

% ==========================================
% 第三部分:深度品读(庄子)
% ==========================================
\section*{\textcolor{philocolor}{两千年前的风声}}

缪尔为了体验风暴,爬上了一棵 100 英尺高的道格拉斯冷杉树顶,在树顶上听到了“金属声”、“流水声”,他发现是树叶的结构决定了风的声音。
早在两千多年前,中国的哲学家庄子也听过这场风。他没有爬树,但他把风的声音分成了三个声部。

% ==========================================
% 第一部分:前奏 · 吾丧我
% ==========================================
\subsection*{\textcolor{stormcolor}{前奏:形如槁木,心如死灰}}

\begin{spacing}{1.3}
\large
南郭子綦(qí)隐几而坐,仰天而嘘,嗒(tà)焉似丧其耦。

颜成子游立侍乎前,曰:“何居乎?形固可使如槁(gǎo)木,而心固可使如死灰乎?今之隐几者,非昔之隐几者也。”

子綦曰:“偃,不亦善乎,而问之也!今者吾丧我,汝知之乎?汝闻人籁而未闻地籁,汝闻地籁而未闻天籁夫!”
\end{spacing}

\begin{tcolorbox}[colback=commentbox, colframe=white, title=\textbf{【夜读微注】}]
\begin{itemize}[leftmargin=*]
    \item \textbf{隐几(yǐn jī)}:靠着小桌子。几,古人跪坐时依靠的小案几。
    \item \textbf{嗒(tà)焉}:失神、忘我的样子。
    \item \textbf{丧其耦(ǒu)}:忘记了自己的身体。耦,指相对的身体或世俗的自我。
    \item \textbf{吾丧我}:这是庄子哲学的核心状态。“吾”(真我)抛弃了“我”(世俗的、充满偏见的我)。
    \item \textbf{画面感}:想象一位老者靠着桌子,仰天长吐一口气,整个人像断了电一样松弛下来,灵魂仿佛出窍了。只有在这样的静止中,才能听到世界的真声。
\end{itemize}
\end{tcolorbox}

% ==========================================
% 第二部分:高潮 · 地籁(大地的交响)
% ==========================================
\subsection*{\textcolor{stormcolor}{高潮:地籁与万窍}}

\begin{spacing}{1.3}
\large
子綦曰:“夫大块噫(yī)气,其名为风。是唯无作,作则万窍怒呺(háo)。而独不闻之翏翏(liù)乎?

山林之畏隹(cuī),大木百围之窍穴,似鼻,似口,似耳,似枅(jī),似圈,似臼,似洼者,似污者。

激者,謞(hè)者,叱者,吸者,叫者,譹(háo)者,宎(yào)者,咬者。前者唱于,而随者唱喁(yú)。

泠(líng)风则小和,飘风则大和,厉风济则众窍为虚。而独不见之调调、之刁刁乎?”
\end{spacing}

\begin{tcolorbox}[colback=commentbox, colframe=white, title=\textbf{【夜读微注】}]
\begin{itemize}[leftmargin=*]
    \item \textbf{大块噫(yī)气}:大自然打了个嗝,吐出一口气,就变成了风。
    \item \textbf{万窍怒呺(háo)}:大地上的千万个孔洞在怒吼。
    \item \textbf{畏隹(wēi cuī)}:形容山林高峻、险怪的样子。
    \item \textbf{声音的模仿(拟声词大赏)}:
        \begin{itemize}
            \item \textbf{激}(像箭射出的声音)、\textbf{謞}(hè,像呼啸声);
            \item \textbf{叱}(像呵斥声)、\textbf{吸}(像呼吸声);
            \item \textbf{叫}(像喊叫)、\textbf{譹}(háo,像哭号);
            \item \textbf{宎}(yào,深沉的声音)、\textbf{咬}(像咬牙切齿的声音)。
        \end{itemize}
    \item \textbf{唱于、唱喁(yú)}:风声前呼后应,像合唱团一样。
    \item \textbf{众窍为虚}:大风一停,所有的孔洞瞬间安静,恢复空虚。
\end{itemize}
\end{tcolorbox}

\vspace{1cm}

% ==========================================
% 第三部分:终章 · 天籁(谁在吹奏?)
% ==========================================
\subsection*{\textcolor{stormcolor}{终章:天籁之问}}

\begin{spacing}{1.3}
\large
子游曰:“地籁则众窍是已,人籁则比竹是已。敢问天籁。”

子綦曰:“夫吹万不同,而使其自己也,咸其自取,怒者其谁邪!”
\end{spacing}

\begin{tcolorbox}[colback=commentbox, colframe=white, title=\textbf{【夜读微注】}]
\begin{itemize}[leftmargin=*]
    \item \textbf{人籁}:人吹排箫(比竹)发出的声音,是人工的。
    \item \textbf{地籁}:风吹孔穴发出的声音,是借力的(风撞击物体)。
    \item \textbf{天籁}:
        \begin{itemize}
            \item \textbf{吹万不同}:风吹过万物,发出的声音各不相同(酸木头发酸声,湿木头发湿声)。
            \item \textbf{使其自己}:让它们展现出自己原本的样子。
            \item \textbf{怒者其谁邪}:最后一句是千古之问——“到底是谁在怒吼?”
        \end{itemize}
\end{itemize}
\end{tcolorbox}

\newpage

\section*{\textcolor{philocolor}{给小朋友的思考题}}

\begin{itemize}
    \item 缪尔爷爷说:橡树发出金属声,是因为叶子硬;冷杉发出流水声,是因为针叶密。这像不像庄子说的“地籁”(众窍)?
    \item 庄子最后问了一个问题:\textbf{“怒者其谁邪?”}(是谁让这些风怒号的呢?)
    \item 既然没有人指挥,风是自己吹的,树是自己响的,那这背后的力量是什么?
\end{itemize}

\end{document}